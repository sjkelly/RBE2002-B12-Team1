% modified from http://www.physics.brocku.ca/doc/sample_lab_report
\documentclass[12pt]{article}
\usepackage{graphicx, natbib}
\addtolength{\textwidth}{1in}
\addtolength{\textheight}{1in}
\addtolength{\evensidemargin}{0.5in}
\addtolength{\oddsidemargin}{-0.5in}
\addtolength{\topmargin}{-0.5in}
%macros
\newcommand{\HRule}{\rule{\linewidth}{0.5mm}}


\begin{document}
    
    %Title and Abstract
    \begin{titlepage}

\begin{center}


% Upper part of the page
\includegraphics{wpi_logo.jpg}\\[1cm]

\textsc{\LARGE Worcester Polytechnic Institute}\\[1cm]

\textsc{\Large Robotics Engineering Program - RBE2002}\\[1cm]


% Title
\HRule \\[1cm]
{ \huge \bfseries Lab 1 : Electrical Circuits and Operational Amplifiers}\\[1cm]

\HRule \\[1cm]

% Author and supervisor
\begin{minipage}{0.4\textwidth}
\begin{flushleft} \large
\emph{Authors:}\\
Dale Herzog \\
Rob Dabrowski \\
Steve Kelly \\
\end{flushleft}
\end{minipage}
\begin{minipage}{0.4\textwidth}
\begin{flushright} \large
\emph{Date:} 
November 7, 2012 
\emph{Instructor:} 
Prof. Putnam 
\emph{Section:} 
B02 - 2012
\end{flushright}
\end{minipage}

\vfill
\end{center}

{ \Large Abstract } \\
    We learned how to...
lksajf\\
 adskjfa \\
akldjfa\\
adskkfjalk\\
akdsjfa\\
akdsjf'a\\

\end{titlepage}

    %End First page
    
    %put table of contents
    \tableofcontents
    \clearpage 

    % Introduction 
    \section{Introduction}
    The purpose of this experiment is to determine 
    xxx xxxxx xxxxx xxx xxxx xxxx xxxxx xxxxx xxxx xxxxx xxxx xxxxxxxxx
    xxx xxxxx xxxxx xxx xxxx xxxx xxxxx xxxxx xxxx xxxxx xxxx xxxxxxxxx
     
    xxx xxxxx xxxxx xxx xxxx xxxx xxxxx xxxxx xxxx xxxxx xxxx xxxxxxxxx
    xxx xxxxx xxxxx xxx xxxx xxxx xxxxx xxxxx xxxx xxxxx xxxx xxxxxxxxx
    xxx xxxxx xxxxx xxx xxxx xxxx~\cite{adams1995hitchhiker} xxxx xxxxx xxxx xxxxxxxxx 
    xxx xxxxx xxxxx xxx xxxx xxxx xxxxx xxxxx xxxx xxxxx xxxx xxxxxxxxx
    This is the way to insert a figure or simply 
    leave some white space for a figure that is to be 
    pasted in later, like a photo or a hand-drawn sketch.  As seen in
    Figure~\ref{fig:picture}, 
    
        %Insert Picture
        \begin{figure}
        \begin{center}
        \includegraphics{wpi_logo.jpg}
        \end{center}
        \caption{This is the caption for the picture.}
        \label{fig:picture}
        \end{figure}
    
    everything is clear.
    xxx xxxxx xxxxx xxx xxxx xxxx xxxxx xxxxx xxxx xxxxx xxxx xxxxxxxxx
    xxx xxxxx xxxxx xxx xxxx xxxx xxxxx xxxxx xxxx xxxxx xxxx xxxxxxxxx
    xxx xxxxx xxxxx xxx xxxx xxxx xxxxx xxxxx xxxx xxxxx xxxx xxxxxxxxx
    
    xxx xxxxx xxxxx xxx xxxx xxxx xxxxx xxxxx xxxx xxxxx xxxx xxxxxxxxx
    xxx xxxxx xxxxx xxx xxxx xxxx xxxxx xxxxx xxxx xxxxx xxxx xxxxxxxxx
    Text before the footnote.\footnote{Here's the text of the footnote.}
    Text after the footnote. xxxx xxxxx xxxx xxxxx xxxxx xxxx xxxxxxxxx
    xxx xxxxx xxxxx xxx xxxx xxxx xxxxx xxxxx xxxx xxxxx xxxx xxxxxxxxx

    %End Introduction
    
    \section{Method}
        \subsection{RLC Circuits}
        \subsection{Operational Amplifiers}
        \subsection{Load Effect}
        \subsection{Amplification}
        \subsection{Filtering}
        \subsection{Operations with Amplifiers}
            \subsubsection{Using the Multisim simulator}
            
            
    \section{Results}
        \subsection{RLC Circuits}
        \subsection{Operational Amplifiers}
        \subsection{Load Effect}
        \subsection{Amplification}
        \subsection{Filtering}
        \subsection{Operations with Amplifiers}
            \subsubsection{Using the Multisim simulator}


    \section{Discussion}
        \subsection{RLC Circuits}
        \subsection{Operational Amplifiers}
        \subsection{Load Effect}
        \subsection{Amplification}
        \subsection{Filtering}
        \subsection{Operations with Amplifiers}
            \subsubsection{Using the Multisim simulator}
    
    \section{Conclusion}
    In conclusion, this experiment 
    xxx xxxxx xxxxx xxx xxxx xxxx xxxxx xxxxx xxxx xxxxx xxxx
    xxxxxxxxx xxx xxxxx xxxxx xxx xxxx xxxx xxxxx xxxxx xxxx xxxxx xxxx xxxxxxxxx
    xxx xxxxx xxxxx xxx xxxx xxxx xxxxx xxxxx xxxx xxxxx xxxx xxxxxxxxx
    xxx xxxxx xxxxx xxx xxxx xxxx xxxxx xxxxx xxxx xxxxx xxxx xxxxxxxxx

    %insert Bibliography
    \bibliographystyle{plain}
    \bibliography{references}

    %start appendix
    \appendix
    \cleardoublepage

    \section{Raw data}
    \label{app:data}
    
    \begin{table}[h]
    \caption{Resistance and Temperature of the Filament}
    \label{tab:data}
    \vspace{0.15in}
    \begin{center}
    \begin{tabular}{|c|c|c|c|}
    \hline
    $R(T)$, $\Omega$ & $T$, K & $1/T$, K$^{-1}$ & $\ln P$ \\
    \hline
    151.00$\pm$3.92 & 828.35$\pm$23.46& $1.2072\times10^{-3}$& -13.29 \\
    157.12$\pm$3.71 & 856.88$\pm$22.25& $1.1671\times10^{-3}$& -12.64 \\
    162.53$\pm$3.49 & 881.99$\pm$21.02& $1.1338\times10^{-3}$& -12.33 \\
    166.67$\pm$3.33 & 901.14$\pm$20.13& $1.1097\times10^{-3}$& -11.90 \\
    171.84$\pm$3.17 & 924.98$\pm$19.25& $1.0811\times10^{-3}$& -11.25 \\ 
    176.84$\pm$3.04 & 947.96$\pm$18.53& $1.0549\times10^{-3}$& -10.77 \\
    181.46$\pm$2.90 & 969.13$\pm$15.49& $1.0319\times10^{-3}$& -10.20 \\
    186.49$\pm$2.79 & 992.09$\pm$17.18& $1.0080\times10^{-3}$& -9.66 \\
    190.91$\pm$2.69 & 1012.21$\pm$16.65& $9.8794\times10^{-4}$& -9.13 \\
    195.48$\pm$2.59 & 1032.95$\pm$16.45& $9.6811\times10^{-4}$& -8.60 \\
    199.93$\pm$2.50 & 1053.08$\pm$15.65& $9.4960\times10^{-4}$& -8.10 \\
    204.47$\pm$2.41 & 1073.56$\pm$15.19& $9.3148\times10^{-4}$& -7.63 \\
    208.62$\pm$2.34 & 1092.22$\pm$14.83& $9.1556\times10^{-4}$& -7.16 \\
    \hline
    \end{tabular}
    \end{center}  
    \end{table}

    \newpage
    \section{A {\tt physica} macro}
    \label{app:macro}
    This {\tt physica} macro was used to generate the plot of Figure
    as well as to fit xxx xxxxx xxxxx xxx xxxx xxxx xxxxx xxxxx xxxx xxxxx xxxx xxxxxxxxx
    xxx xxxxx xxxxx xxx xxxx xxxx xxxxx xxxxx xxxx xxxxx xxxx xxxxxxxxx
    
    {\tt 
    \begin{verbatim}
    
          ! exp_3.pcm
          clear
          
          ! read in the data
          read\format\noerror exp_3.dat (*) x,y,dy
    
          ! plot the data
          label\x `Voltage, V'
          label\y `Power, W'
          set colour 1 1
          set pchar -4
          graph x,y,dy
    
          ! fit and plot the curve
          scalar\vary A,T,w,phi
          ! initial values for parameters
          A=2.3
          w=6.5
          phi=0
          T=10.
    
          fit y=A*cos(w*x+phi)*exp(-x**2/T)
          fit\update f
          set colour 2 2
          set pchar 0
          graph\noaxes x,f
    
    \end{verbatim}
    }
\end{document}

